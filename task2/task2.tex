\documentclass[12pt, a4paper]{article}
\usepackage[utf8x]{inputenc}
\usepackage{amsmath}
\usepackage[russian]{babel}
\usepackage[width=18cm, top=2cm, bottom=2cm]{geometry}
\usepackage{indentfirst}
\usepackage{enumerate}
\usepackage{amsfonts}
\usepackage{amssymb}
\usepackage{graphicx}
\usepackage{multirow}
\usepackage{pgf}
\usepackage{pgfplotstable}
		
		\title{Задание №2\\
			Интерполяция функций полиномами}
		\author{Збродов Владислав}
		
\begin{document}
	\maketitle
	Необходимо выяснить, что будет происходить при росте числа интервалов интерполяции в случае использования равномерных и чебышевских сеток.
	\section{Графики}
	
	\subsection{$f(x) = \frac{1}{1+x^2}, N = 2$}
	\begin{tikzpicture}
		\begin{axis}[
			width=16cm, height=7cm,
			grid=major, legend pos=south east,
			xlabel={$x$}, ylabel={$f(x)$},ymin=0
			]
			\node at (axis cs:5,0.9){\color{red}Эталон};
			\node at (axis cs:5,0.7){\color{green}Равномерная сетка};
			\node at (axis cs:5,0.5){\color{blue}Чебышевская сетка};
			\addplot[color=green,domain=-6:6,samples=100] table{out/tex/res_uniform.dat};
			\addplot[color=blue,domain=-6:6,samples=100] table{out/tex/res_chebyshev.dat};
			\addplot[color=red,domain=-6:6,samples=100]{1/(1+x*x)};
			%\legend{$x = \overline{-6, 6}, \, y = 7, 3, 12, 4, 9, 1$};
		\end{axis}
	\end{tikzpicture}
		\subsection{$f(x) = \frac{1}{1+x^2}, N = 4$}
	\begin{tikzpicture}
		\begin{axis}[
			width=16cm, height=7cm,
			grid=major, legend pos=south east,
			xlabel={$x$}, ylabel={$f(x)$}, ymin=0
			]
			\node at (axis cs:5,0.9){\color{red}Эталон};
			\node at (axis cs:5,0.7){\color{green}Равномерная сетка};
			\node at (axis cs:5,0.5){\color{blue}Чебышевская сетка};
			\addplot[color=green,domain=-6:6,samples=100] table{out/tex/res_uniform1.dat};
			\addplot[color=blue,domain=-6:6,samples=100] table{out/tex/res_chebyshev1.dat};
			\addplot[color=red,domain=-6:6,samples=100]{1/(1+x*x)};
			%\legend{$x = \overline{-6, 6}, \, y = 7, 3, 12, 4, 9, 1$};
		\end{axis}
	\end{tikzpicture}
		\subsection{$f(x) = \frac{1}{1+x^2}, N = 8$}
	\begin{tikzpicture}
		\begin{axis}[
			width=16cm, height=7cm,
			grid=major, legend pos=south east,
			xlabel={$x$}, ylabel={$f(x)$}, ymin=0
			]
			\node at (axis cs:5,0.9){\color{red}Эталон};
			\node at (axis cs:5,0.7){\color{green}Равномерная сетка};
			\node at (axis cs:5,0.5){\color{blue}Чебышевская сетка};
			\addplot[color=green,domain=-6:6,samples=100] table{out/tex/res_uniform2.dat};
			\addplot[color=blue,domain=-6:6,samples=100] table{out/tex/res_chebyshev2.dat};
			\addplot[color=red,domain=-6:6,samples=100]{1/(1+x*x)};
			%\legend{$x = \overline{-6, 6}, \, y = 7, 3, 12, 4, 9, 1$};
		\end{axis}
	\end{tikzpicture}
		\subsection{$f(x) = \frac{1}{1+x^2}, N = 16$}
	\begin{tikzpicture}
		\begin{axis}[
			width=16cm, height=7cm,
			grid=major, legend pos=south east,
			xlabel={$x$}, ylabel={$f(x)$}, ymin=0,ymax=1.2
			]
			\node at (axis cs:5,0.9){\color{red}Эталон};
			\node at (axis cs:5,0.7){\color{green}Равномерная сетка};
			\node at (axis cs:5,0.5){\color{blue}Чебышевская сетка};
			\addplot[color=green,domain=-6:6,samples=100] table{out/tex/res_uniform3.dat};
			\addplot[color=blue,domain=-6:6,samples=100] table{out/tex/res_chebyshev3.dat};
			\addplot[color=red,domain=-6:6,samples=100]{1/(1+x*x)};
			%\legend{$x = \overline{-6, 6}, \, y = 7, 3, 12, 4, 9, 1$};
		\end{axis}
	\end{tikzpicture} \\
	При увеличении $N$, равномерная сетка в окрестности нуля справляется лучше, чем чебышевская, но на краях всё становится ужасно.
	Чебышевская сетка справляется  отлично на краях отрезка, 
	\subsection{$N=2$ \begin{equation*}
			f(x) = 
			\begin{cases}
				1 &\text{ $x = 1$}\\
				0 &\text{ $x \ne 1$}
			\end{cases}
	\end{equation*}}
	\begin{tikzpicture}
		\begin{axis}[
			width=16cm, height=7cm,
			grid=major, legend pos=south east,
			xlabel={$x$}, ylabel={$f(x)$}, ymin=0
			]
			\node at (axis cs:5,0.7){\color{green}Равномерная сетка};
			\node at (axis cs:5,0.5){\color{blue}Чебышевская сетка};
			\addplot[color=green,domain=-6:6,samples=100] table{out/tex/res_uniform4.dat};
			\addplot[color=blue,domain=-6:6,samples=100] table{out/tex/res_chebyshev4.dat};
			%\legend{$x = \overline{-6, 6}, \, y = 7, 3, 12, 4, 9, 1$};
		\end{axis}
	\end{tikzpicture}
		\subsection{$N=4$ \begin{equation*}
			f(x) = 
			\begin{cases}
				1 &\text{ $x = 0$}\\
				0 &\text{ $x \ne 0$}
			\end{cases}
	\end{equation*}}
	\begin{tikzpicture}
		\begin{axis}[
			width=16cm, height=7cm,
			grid=major, legend pos=south east,
			xlabel={$x$}, ylabel={$f(x)$}, ymin=-0.2
			]
			\node at (axis cs:5,0.7){\color{green}Равномерная сетка};
			\node at (axis cs:5,0.5){\color{blue}Чебышевская сетка};
			\addplot[color=green,domain=-6:6,samples=100] table{out/tex/res_uniform5.dat};
			\addplot[color=blue,domain=-6:6,samples=100] table{out/tex/res_chebyshev5.dat};
			%\legend{$x = \overline{-6, 6}, \, y = 7, 3, 12, 4, 9, 1$};
		\end{axis}
	\end{tikzpicture}
		\subsection{$N=8$ \begin{equation*}
			f(x) = 
			\begin{cases}
				1 &\text{ $x = 0$}\\
				0 &\text{ $x \ne 0$}
			\end{cases}
	\end{equation*}}
	\begin{tikzpicture}
		\begin{axis}[
			width=16cm, height=7cm,
			grid=major, legend pos=south east,
			xlabel={$x$}, ylabel={$f(x)$}, ymin=-0.2
			]
			\node at (axis cs:5,0.7){\color{green}Равномерная сетка};
			\node at (axis cs:5,0.5){\color{blue}Чебышевская сетка};
			\addplot[color=green,domain=-6:6,samples=100] table{out/tex/res_uniform6.dat};
			\addplot[color=blue,domain=-6:6,samples=100] table{out/tex/res_chebyshev6.dat};
			%\legend{$x = \overline{-6, 6}, \, y = 7, 3, 12, 4, 9, 1$};
		\end{axis}
	\end{tikzpicture} 
		\subsection{$N=16$ \begin{equation*}
			f(x) = 
			\begin{cases}
				1 &\text{ $x = 0$}\\
				0 &\text{ $x \ne 0$}
			\end{cases}
	\end{equation*}}
	\begin{tikzpicture}
		\begin{axis}[
			width=16cm, height=7cm,
			grid=major, legend pos=south east,
			xlabel={$x$}, ylabel={$f(x)$}, ymin=-0.2
			]
			\node at (axis cs:5,7){\color{green}Равномерная сетка};
			\node at (axis cs:5,5){\color{blue}Чебышевская сетка};
			\addplot[color=green,domain=-6:6,samples=100] table{out/tex/res_uniform7.dat};
			\addplot[color=blue,domain=-6:6,samples=100] table{out/tex/res_chebyshev7.dat};
			%\legend{$x = \overline{-6, 6}, \, y = 7, 3, 12, 4, 9, 1$};
		\end{axis}
	\end{tikzpicture} \\
	Оба способа интерполяции выглядят не очень хорошо, но сетка Чебышева проявляет
	себя гораздо лучше хотя бы в окрестностях концов отрезка.
	\newpage
	
	\section{Проверка на нечетной функции. Исправление зеркальности}
	\textbf{До исправления:}\\
	\begin{tikzpicture}
		\begin{axis}[
			width=16cm, height=7cm,
			grid=major, legend pos=south east,
			xlabel={$x$}, ylabel={$f(x)$}
			]
			\node at (axis cs:0,200){\color{red}Эталон};
			\node at (axis cs:0,160){\color{blue}Равномерная сетка};
			\node at (axis cs:0,120){\color{green}Чебышевская сетка};
			\addplot[color=green,domain=-6:6,samples=100] table{out/test/res_chebyshev.dat};
			\addplot[color=blue,domain=-6:6,samples=100] table{out/test/res_uniform.dat};
			\addplot[color=red,domain=-6:6,samples=100]{(x*x*x)};
			%\legend{$x = \overline{-6, 6}, \, y = 7, 3, 12, 4, 9, 1$};
		\end{axis}
	\end{tikzpicture}
	
		\textbf{После исправления:}\\
	\begin{tikzpicture}
		\begin{axis}[
			width=16cm, height=7cm,
			grid=major, legend pos=south east,
			xlabel={$x$}, ylabel={$f(x)$}
			]
			\node at (axis cs:0,200){\color{red}Эталон};
			\node at (axis cs:0,160){\color{blue}Равномерная сетка};
			\node at (axis cs:0,120){\color{green}Чебышевская сетка};
			\addplot[color=green,domain=-6:6,samples=100] table{out/test/res_chebyshev1.dat};
			\addplot[color=blue,domain=-6:6,samples=100] table{out/test/res_uniform1.dat};
			\addplot[color=red,domain=-6:6,samples=100]{(x*x*x)};
			%\legend{$x = \overline{-6, 6}, \, y = 7, 3, 12, 4, 9, 1$};
		\end{axis}
	\end{tikzpicture}
	
	
	
\end{document}